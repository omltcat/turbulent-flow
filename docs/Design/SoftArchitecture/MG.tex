\documentclass[12pt, titlepage]{article}

\usepackage{fullpage}
\usepackage[round]{natbib}
\usepackage{multirow}
\usepackage{booktabs}
\usepackage{tabularx}
\usepackage{graphicx}
\usepackage{float}
\usepackage{hyperref}
\hypersetup{
    colorlinks,
    citecolor=blue,
    filecolor=black,
    linkcolor=red,
    urlcolor=blue
}

\input{../../Comments}
%% Common Parts

\newcommand{\progname}{SynthEddy} % PUT YOUR PROGRAM NAME HERE
\newcommand{\authname}{Phil Du (Software)
\\ Nikita Holyev (Theory)
\\ Dr. Spencer Smith (Supervisor)
} % AUTHOR NAMES                  

\usepackage{hyperref}
    \hypersetup{colorlinks=true, linkcolor=blue, citecolor=blue, filecolor=blue,
                urlcolor=blue, unicode=false}
    \urlstyle{same}
                                


\newcounter{acnum}
\newcommand{\actheacnum}{AC\theacnum}
\newcommand{\acref}[1]{AC\ref{#1}}

\newcounter{ucnum}
\newcommand{\uctheucnum}{UC\theucnum}
\newcommand{\uref}[1]{UC\ref{#1}}

\newcounter{mnum}
\newcommand{\mthemnum}{M\themnum}
\newcommand{\mref}[1]{M\ref{#1}}

\begin{document}

\title{Module Guide for \progname{}} 
\author{\authname}
\date{\today}

\maketitle

\pagenumbering{roman}

\section{Revision History}

\begin{tabularx}{\textwidth}{p{3cm}p{2cm}X}
\toprule {\bf Date} & {\bf Version} & {\bf Notes}\\
\midrule
2024-03-18 & 1.0 & Initial MG\\
2024-03-21 & 1.1 & Feedbacks by domain expert addressed\\
2024-04-15 & 1.2 & Changes to module designs due to performance concerns\\
2024-09-06 & 1.3 & Add documentation for field wrap-around and chunking\\
% Date 2 & 1.1 & Notes\\
\bottomrule
\end{tabularx}

\newpage

\section{Reference Material}

This section records information for easy reference.

\subsection{Abbreviations and Acronyms}

\renewcommand{\arraystretch}{1.2}
\begin{tabular}{l l} 
  \toprule		
  \textbf{symbol} & \textbf{description}\\
  \midrule 
  AC & Anticipated Change\\
  CFD & Computational Fluid Dynamics\\
  DAG & Directed Acyclic Graph \\
  GUI & Graphical User Interface\\
  I/O & Input/Output\\
  M & Module \\
  MG & Module Guide \\
  NFR & Non-Functional Requirement\\
  NumPy &  Python Package for Scientific Computing\\
  OS & Operating System \\
  R & Requirement\\
  SC & Scientific Computing \\
  SRS & Software Requirements Specification\\
  \progname & Synthetic Turbulent Flow Generator\\
  UC & Unlikely Change \\
  \bottomrule
\end{tabular}\\

\newpage

\tableofcontents

\listoftables

\listoffigures

\newpage

\pagenumbering{arabic}

\section{Introduction}

Decomposing a system into modules is a commonly accepted approach to developing
software.  A module is a work assignment for a programmer or programming
team~\citep{ParnasEtAl1984}.  We advocate a decomposition
based on the principle of information hiding~\citep{Parnas1972a}.  This
principle supports design for change, because the ``secrets'' that each module
hides represent likely future changes.  Design for change is valuable in SC,
where modifications are frequent, especially during initial development as the
solution space is explored.  

Our design follows the rules layed out by \citet{ParnasEtAl1984}, as follows:
\begin{itemize}
\item System details that are likely to change independently should be the
  secrets of separate modules.
\item Each data structure is implemented in only one module.
\item Any other program that requires information stored in a module's data
  structures must obtain it by calling access programs belonging to that module.
\end{itemize}

After completing the first stage of the design, the Software Requirements
Specification (SRS), the Module Guide (MG) is developed~\citep{ParnasEtAl1984}. The MG
specifies the modular structure of the system and is intended to allow both
designers and maintainers to easily identify the parts of the software.  The
potential readers of this document are as follows:

\begin{itemize}
\item New project members: This document can be a guide for a new project member
  to easily understand the overall structure and quickly find the
  relevant modules they are searching for.
\item Maintainers: The hierarchical structure of the module guide improves the
  maintainers' understanding when they need to make changes to the system. It is
  important for a maintainer to update the relevant sections of the document
  after changes have been made.
\item Designers: Once the module guide has been written, it can be used to
  check for consistency, feasibility, and flexibility. Designers can verify the
  system in various ways, such as consistency among modules, feasibility of the
  decomposition, and flexibility of the design.
\end{itemize}

The rest of the document is organized as follows. Section
\ref{SecChange} lists the anticipated and unlikely changes of the software
requirements. Section \ref{SecMH} summarizes the module decomposition that
was constructed according to the likely changes. Section \ref{SecConnection}
specifies the connections between the software requirements and the
modules. Section \ref{SecMD} gives a detailed description of the
modules. Section \ref{SecTM} includes two traceability matrices. One checks
the completeness of the design against the requirements provided in the SRS. The
other shows the relation between anticipated changes and the modules. Section
\ref{SecUse} describes the use relation between modules.

\section{Anticipated and Unlikely Changes} \label{SecChange}

This section lists possible changes to the system. According to the likeliness
of the change, the possible changes are classified into two
categories. Anticipated changes are listed in Section \ref{SecAchange}, and
unlikely changes are listed in Section \ref{SecUchange}.

\subsection{Anticipated Changes} \label{SecAchange}

Anticipated changes are the source of the information that is to be hidden
inside the modules. Ideally, changing one of the anticipated changes will only
require changing the one module that hides the associated decision. The approach
adapted here is called design for
change.

\begin{description}
\item[\refstepcounter{acnum} \actheacnum \label{acRealEddy}:] User can input some hyperparameters to generate realistic eddy profiles instead needing manual profile input. [SRS: LC1].
\item[\refstepcounter{acnum} \actheacnum \label{acInternal}:] Allow generating internal flow instead of external [SRS: LC2].
\item[\refstepcounter{acnum} \actheacnum \label{acQuery}:] Adding more query protocols/endpoints for integration with CFD software [SRS: R3].
\item[\refstepcounter{acnum} \actheacnum \label{acShape}:] Adding shape functions to choose from, potentially done by user [SRS: NFR3].
\item[\refstepcounter{acnum} \actheacnum \label{acVisual}:] Output 3D or 2D cross-section visualization of flow field.
\end{description}

\subsection{Unlikely Changes} \label{SecUchange}

The module design should be as general as possible. However, a general system is
more complex. Sometimes this complexity is not necessary. Fixing some design
decisions at the system architecture stage can simplify the software design. If
these decision should later need to be changed, then many parts of the design
will potentially need to be modified. Hence, it is not intended that these
decisions will be changed.

\begin{description}
\item[\refstepcounter{ucnum} \uctheucnum \label{uc2D}:] Work with 2D flow field instead of 3D [SRS: LC3].
\end{description}

\section{Module Hierarchy} \label{SecMH}

This section provides an overview of the module design. Modules are summarized
in a hierarchy decomposed by secrets in Table \ref{TblMH}. The modules listed
below, which are leaves in the hierarchy tree, are the modules that will
actually be implemented.

\begin{description}
\item [\refstepcounter{mnum} \mthemnum \label{mHH}:] Hardware-Hiding Module
\item [\refstepcounter{mnum} \mthemnum \label{mMain}:] Main Control Module
\item [\refstepcounter{mnum} \mthemnum \label{mQuery}:] Query Interface
\item [\refstepcounter{mnum} \mthemnum \label{mProf}:] Eddy Profile Module
\item [\refstepcounter{mnum} \mthemnum \label{mFlow}:] Flow Field Module
\item [\refstepcounter{mnum} \mthemnum \label{mEddy}:] Eddy Module
\item [\refstepcounter{mnum} \mthemnum \label{mShape}:] Shape Function Module
\item [\refstepcounter{mnum} \mthemnum \label{mFile}:] File I/O Module
\item [\refstepcounter{mnum} \mthemnum \label{mVector}:] Vector Module
\item [\refstepcounter{mnum} \mthemnum \label{mVisual}:] Visualization Module (Placeholder)
\end{description}


\begin{table}[h!]
\centering
\begin{tabular}{p{0.3\textwidth} p{0.6\textwidth}}
\toprule
\textbf{Level 1} & \textbf{Level 2}\\
\midrule

{Hardware-Hiding Module} & ~ \\
\midrule

\multirow{6}{0.3\textwidth}{Behaviour-Hiding Module}
& Main Control Module\\
& Query Interface\\
& Eddy Profile Module\\
& Flow Field Module\\
& Eddy Module\\
& Shape Function Module\\
\midrule

\multirow{3}{0.3\textwidth}{Software Decision Module} 
& File I/O Module\\
& Vector Module\\
& Visualization Module\\
\bottomrule

\end{tabular}
\caption{Module Hierarchy}
\label{TblMH}
\end{table}

\section{Connection Between Requirements and Design} \label{SecConnection}

The design of the system is intended to satisfy the requirements developed in
the SRS. In this stage, the system is decomposed into modules. The connection
between requirements and modules is listed in Table~\ref{TblRT}.

To facilitate the transformation from mathematical models to a usable software, some design decisions are made, which are not part of the models and not explicitly mentioned in the SRS. These are documented as follows:

\subsection{Field Wrap-around}
Eddies that are touching the boundary of the field would have some flow out-of and into the field. To maintain conservation of mass mandated in the SRS, these flows need to be accounted for and compensated. A common practice is to move the part of an eddy that is outside the field to the opposite side of the field (wrap-around). However, if there is field is flowing with a uniform average velocity, this will introduce a situation where the field is repeating itself with predictable periodicity. This would not be a realistic representation of a turbulent flow field. To address this, the following designs are used, depending on the flow velocity and pattern:

\subsubsection{Stationary Field: Classic Wrap-around}
If the user inputs a zero average $x$-velocity, the classic wrap-around as described above is used. For example, if an eddy is protruding out from the positive $x$-boundary, the part that is outside should be moved to protrude in at the negative $x$-boundary, with the same $y$ and $z$ coordinates. This does mean that the field repeats itself in any direction. This decision was made after discussion with the domain expert as zero average velocity is more of a research scenario, in which this simple wrap-around is more desirable for related research computations.

To achieve this wrap around effect, the entire field is copied and shifted in the opposite direction. i.e. to wrap around the positive $x$-boundary, the entire field is copied and placed outside the negative $x$-boundary of the original field, so that the negative $x$-boundary of the original field is mated with the positive $x$-boundary of the copied field. This is done for all boundaries, and also diagonally and at the corners, so that any protruding eddies are wrapped around to the opposite side. This results in a field that is 27 times the original field size, with the original field at the center. To avoid wasting memory, all the eddies in the copied field that are not touching the original field boundary are discarded.

\subsubsection{Non-stationary Field: Random Wrap-around in flow direction}
If the user inputs a non-zero average $x$-velocity that is uniform across the entire field, the classic wrap-around would result in a periodic flow field. To avoid this, instead of the classic wrap-around in $x$-direction, a "flow iteration" mechanism is introduced. When the field is generated, two more fields (iterations) outside the positive and negative $x$-boundaries are also generated. These additional fields have eddies with same strength and directions, at same $x$ coordinates as those in the original field, but different and random $y$ and $z$ coordinates. As the flow moves forward, additional fields are randomly generated and attached to the end, like frames on a film reel. This ensures that when viewing at any section on this "reel" with the same $x$ width as the original field, the amount of eddies is always the same, and any eddies protruding out the $x$-boundaries are accounted for by eddies in the next/previous iteration, but at different $y$ and $z$ coordinates to avoid periodicity. Wrap-around at the $y$ and $z$ boundaries are still done as in the classic wrap-around.

\subsubsection{Non-stationary Field with Non-uniform Average Velocity}
If the user inputs a non-zero average $x$-velocity that is not uniform across the entire field, but rather as a function of $y$ and $z$, each velocity that each eddy center moves would be different depending on its $y$ and $z$ coordinates. The flow iteration mechanism cannot be applied here as it relies on everything moving at the same pace. Instead, the classic wrap-around is again used.

This design decision is made because, although for each individual eddy, its reappearance at the same location is predictable, when looking at the entire field with many eddies moving at different $x$-velocities, the field as a whole would not repeat itself like with uniform average velocity.

\subsection{Chunking}

While the instance model (IM1 from SRS) requires that to calculate velocity at any location, the influence of all eddies need to be considered. However, most shape functions (such as TM1) sets a cutoff distance. This means that the influence of most eddies are in fact zero. To avoid unnecessary computation of the distance and influence of each eddies against each mesh grid point in the field, the field is divided into chunks of cubic shape. Only eddies that are either within or outside but touching a chunk are considered for calculations against locations inside that chunk. This filtering requires only simple checks of $x$, $y$, and $z$ coordinates of the eddy center against the chunk boundaries, which is much faster than calculating the distance between the eddy center and each mesh grid point.

While one eddy can influence multiple mesh grid points, each point is independent of each other. This means the chunks can potentially be processed in parallel to speed up the computation.

% \wss{The intention of this section is to document decisions that are made
%   ``between'' the requirements and the design.  To satisfy some requirements,
%   design decisions need to be made.  Rather than make these decisions implicit,
%   they are explicitly recorded here.  For instance, if a program has security
%   requirements, a specific design decision may be made to satisfy those
%   requirements with a password.}

\section{Module Decomposition} \label{SecMD}

Modules are decomposed according to the principle of ``information hiding''
proposed by \citet{ParnasEtAl1984}. The \emph{Secrets} field in a module
decomposition is a brief statement of the design decision hidden by the
module. The \emph{Services} field specifies \emph{what} the module will do
without documenting \emph{how} to do it. For each module, a suggestion for the
implementing software is given under the \emph{Implemented By} title. If the
entry is \emph{OS}, this means that the module is provided by the operating
system or by standard programming language libraries.  \emph{\progname{}} means the
module will be implemented by the \progname{} software.

Only the leaf modules in the hierarchy have to be implemented. If a dash
(\emph{--}) is shown, this means that the module is not a leaf and will not have
to be implemented.

\subsection{Hardware Hiding Modules (\mref{mHH})}

\begin{description}
\item[Secrets:]The data structure and algorithm used to implement the virtual
  hardware.
\item[Services:]Serves as a virtual hardware used by the rest of the
  system. This module provides the interface between the hardware and the
  software. So, the system can use it to display outputs or to accept inputs.
\item[Implemented By:] OS
\end{description}

\subsection{Behaviour-Hiding Module}

\begin{description}
\item[Secrets:]The contents of the required behaviours.
\item[Services:]Includes programs that provide externally visible behaviour of
  the system as specified in the software requirements specification (SRS)
  documents. This module serves as a communication layer between the
  hardware-hiding module and the software decision module. The programs in this
  module will need to change if there are changes in the SRS.
\item[Implemented By:] --
\end{description}

\subsubsection{Main Control Module (\mref{mMain})}
\begin{description}
\item[Secrets:]Overall flow of the program
\item[Services:]Taking command line arguments, initializing the system, and calling appropriate modules.
\item[Implemented By:] \progname{}
\item[Type of Module:] Abstract Object
\end{description}

\subsubsection{Query Interface (\mref{mQuery})}
\begin{description}
\item[Secrets:]Query format and structure.
\item[Services:]Provide endpoints for manual and automated query. Unpack query and pass to Flow Field Module for processing. Serialize and return response.
\item[Implemented By:] \progname{}
\item[Type of Module:] Abstract Object
% [Record, Library, Abstract Object, or Abstract Data Type]
  % [Information to include for leaf modules in the decomposition by secrets tree.]
\end{description}

\subsubsection{Eddy Profile Module (\mref{mProf})}
\begin{description}
\item[Secrets:]Algorithm to generate eddy profiles.
\item[Services:]Generate physically realistic eddy profile based on hyperparameters (not implemented currently) or load existing profile.
\item[Implemented By:] \progname{}
\item[Type of Module:] Abstract Data Type
\end{description}

\subsubsection{Flow Field Module (\mref{mFlow})}
\begin{description}
\item[Secrets:]Mechanisms to ensure conservation of mass within flow field.
\item[Services:]Initialize flow field. Compute velocity vector at any given point in the flow field at any given time.
\item[Implemented By:] \progname{}
\item[Type of Module:] Abstract Data Type
\end{description}

\subsubsection{Eddy Module (\mref{mEddy})}
\begin{description}
\item[Secrets:]Mathematical model of velocity around an eddy center, given a shape function.
\item[Services:]Compute the velocity vector at any given point relative to the eddy center.
\item[Implemented By:] \progname{}
\item[Type of Module:] Library
\end{description}

\subsubsection{Shape Function Module (\mref{mShape})}
\begin{description}
\item[Secrets:]Shape function equations
\item[Services:]Providing a list of shape functions to choose from. Allow setting currently active shape function and cutoff value to use.
\item[Implemented By:] \progname{}
\item[Type of Module:] Abstract Object
\end{description}


\subsection{Software Decision Module}

\begin{description}
\item[Secrets:] The design decision based on mathematical theorems, physical
  facts, or programming considerations. The secrets of this module are
  \emph{not} described in the SRS.
\item[Services:] Includes data structure and algorithms used in the system that
  do not provide direct interaction with the user. 
  % Changes in these modules are more likely to be motivated by a desire to
  % improve performance than by externally imposed changes.
\item[Implemented By:] --
\end{description}

\subsubsection{File I/O Module (\mref{mFile})}
\begin{description}
\item[Secrets:]File format and structure
\item[Services:]Read and write to file for persistent eddy profiles and flow fields between program runs.
\item[Implemented By:] \progname{}
\item[Type of Module:] Library
\end{description}

\subsubsection{Vector Module (\mref{mVector})}
\begin{description}
\item[Secrets:]Algorithms for fast vector operations
\item[Services:]Common vector/matrix operations.
\item[Implemented By:] NumPy
\item[Type of Module:] Abstract Data Type
\end{description}

\subsubsection{Visualization Module (\mref{mVisual})}
\begin{description}
\item[Secrets:]Placeholder, no consideration yet.
\item[Services:]Render visualization of flow field.
\item[Implemented By:] \progname{}
\item[Type of Module:] Library
\end{description}

\section{Traceability Matrix} \label{SecTM}

This section shows two traceability matrices: between the modules and the
requirements and between the modules and the anticipated changes.

% the table should use mref, the requirements should be named, use something
% like fref
\begin{table}[H]
\centering
\begin{tabular}{p{0.2\textwidth} p{0.6\textwidth}}
\toprule
\textbf{Req.} & \textbf{Modules}\\
\midrule
R1 & \mref{mQuery}, \mref{mFlow}\\
R2 & \mref{mFlow}, \mref{mEddy}\\
R3 & \mref{mQuery}\\
NFR1 & \mref{mProf}, \mref{mFlow}\\
NFR2 & \mref{mMain}, \mref{mQuery}\\
NFR3 & \mref{mShape}\\
NFR4 & \mref{mHH}\\
\bottomrule
\end{tabular}
\caption{Trace Between Requirements and Modules}
\label{TblRT}
\end{table}

\begin{table}[H]
\centering
\begin{tabular}{p{0.2\textwidth} p{0.6\textwidth}}
\toprule
\textbf{AC} & \textbf{Modules}\\
\midrule
\acref{acRealEddy} & \mref{mProf}\\
\acref{acInternal} & \mref{mFlow}\\
\acref{acQuery} & \mref{mQuery}\\
\acref{acShape} & \mref{mShape}\\
\acref{acVisual} & \mref{mVisual}\\
\end{tabular}
\caption{Trace Between Anticipated Changes and Modules}
\label{TblACT}
\end{table}

\section{Use Hierarchy Between Modules} \label{SecUse}

In this section, the uses hierarchy between modules is
provided. \citet{Parnas1978} said of two programs A and B that A {\em uses} B if
correct execution of B may be necessary for A to complete the task described in
its specification. That is, A {\em uses} B if there exist situations in which
the correct functioning of A depends upon the availability of a correct
implementation of B.  Figure \ref{FigUH} illustrates the use relation between
the modules. It can be seen that the graph is a directed acyclic graph
(DAG). Each level of the hierarchy offers a testable and usable subset of the
system, and modules in the higher level of the hierarchy are essentially simpler
because they use modules from the lower levels.

\begin{figure}[H]
\centering
\includegraphics[width=0.6\textwidth]{UsesHierarchy.png}
\caption{Use hierarchy among modules}
\label{FigUH}
\end{figure}

%\section*{References}

\section{User Interfaces}
N/A (No GUI)
% \wss{Design of user interface for software and hardware.  Attach an appendix if
% needed. Drawings, Sketches, Figma}

\section{Design of Communication Protocols}
Potentially use TCP socket or HTTP (RESTful API) for \acref{acQuery}. This will largely dependent on the CFD side.

% \wss{If appropriate}

% \section{Timeline}

% \wss{Schedule of tasks and who is responsible}

% \wss{You can point to GitHub if this information is included there}

\bibliographystyle {plainnat}
\bibliography{../../../refs/References}

\newpage{}

\end{document}