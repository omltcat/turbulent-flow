\documentclass[12pt, titlepage]{article}

\usepackage{amsmath, mathtools}

\usepackage[round]{natbib}
\usepackage{amsfonts}
\usepackage{amssymb}
\usepackage{graphicx}
\usepackage{colortbl}
\usepackage{xr}
\usepackage{hyperref}
\usepackage{longtable}
\usepackage{xfrac}
\usepackage{tabularx}
\usepackage{float}
\usepackage{siunitx}
\usepackage{booktabs}
\usepackage{multirow}
\usepackage[section]{placeins}
\usepackage{caption}
\usepackage{fullpage}

\hypersetup{
bookmarks=true,     % show bookmarks bar?
colorlinks=true,       % false: boxed links; true: colored links
linkcolor=red,          % color of internal links (change box color with linkbordercolor)
citecolor=blue,      % color of links to bibliography
filecolor=magenta,  % color of file links
urlcolor=cyan          % color of external links
}

\usepackage{array}

\externaldocument{../../SRS/SRS}

%% Comments

\usepackage{color}

\newif\ifcomments\commentstrue %displays comments
%\newif\ifcomments\commentsfalse %so that comments do not display

\ifcomments
\newcommand{\authornote}[3]{\textcolor{#1}{[#3 ---#2]}}
\newcommand{\todo}[1]{\textcolor{red}{[TODO: #1]}}
\else
\newcommand{\authornote}[3]{}
\newcommand{\todo}[1]{}
\fi

\newcommand{\wss}[1]{\authornote{blue}{SS}{#1}} 
\newcommand{\plt}[1]{\authornote{magenta}{TPLT}{#1}} %For explanation of the template
\newcommand{\an}[1]{\authornote{cyan}{Author}{#1}}

%% Common Parts

\newcommand{\progname}{ProgName} % PUT YOUR PROGRAM NAME HERE
\newcommand{\authname}{Team \#, Team Name
\\ Student 1 name
\\ Student 2 name
\\ Student 3 name
\\ Student 4 name} % AUTHOR NAMES                  

\usepackage{hyperref}
    \hypersetup{colorlinks=true, linkcolor=blue, citecolor=blue, filecolor=blue,
                urlcolor=blue, unicode=false}
    \urlstyle{same}
                                


\begin{document}

\title{Module Interface Specification for \progname{}}

\author{\authname}

\date{\today}

\maketitle

\pagenumbering{roman}

\section{Revision History}

\begin{tabularx}{\textwidth}{p{3cm}p{2cm}X}
\toprule {\bf Date} & {\bf Version} & {\bf Notes}\\
\midrule
2024-03-18 & 1.0 & Initial MIS\\
2024-03-21 & 1.1 & Feedbacks by domain expert addressed\\
% Date 2 & 1.1 & Notes\\
\bottomrule
\end{tabularx}

~\newpage

\section{Symbols, Abbreviations and Acronyms}

See SRS Documentation at \href{https://github.com/omltcat/turbulent-flow/blob/main/docs/SRS/SRS.pdf}{GitHub repo}

% \wss{Also add any additional symbols, abbreviations or acronyms}

\newpage

\tableofcontents

\newpage

\pagenumbering{arabic}

\section{Introduction}

The following document details the Module Interface Specifications for \progname{}, a software to artificially generate flow field that mimics turbulent flow, which can be use as starting point for CFD simulation.
% \wss{Fill in your project name and description}

Complementary documents include the System Requirement Specifications
and Module Guide.  The full documentation and implementation can be
found at \href{https://github.com/omltcat/turbulent-flow/blob/main/docs/SRS/SRS.pdf}{SRS}, \href{https://github.com/omltcat/turbulent-flow/blob/main/docs/Design/SoftArchitecture/MG.pdf}{MG}.  
% \wss{provide the url for your repo}

\section{Notation}

% \wss{You should describe your notation.  You can use what is below as
  % a starting point.}

The structure of the MIS for modules comes from \citet{HoffmanAndStrooper1995},
with the addition that template modules have been adapted from
\cite{GhezziEtAl2003}.  The mathematical notation comes from Chapter 3 of
\citet{HoffmanAndStrooper1995}.  For instance, the symbol := is used for a
multiple assignment statement and conditional rules follow the form $(c_1
\Rightarrow r_1 | c_2 \Rightarrow r_2 | ... | c_n \Rightarrow r_n )$.

The following table summarizes the primitive data types used by \progname. 

\begin{center}
\renewcommand{\arraystretch}{1.2}
\noindent 
\begin{tabular}{l l p{7.5cm}} 
\toprule 
\textbf{Data Type} & \textbf{Notation} & \textbf{Description}\\ 
\midrule
character & char & a single symbol or digit\\
integer & $\mathbb{Z}$ & a number without a fractional component in (-$\infty$, $\infty$) \\
natural number & $\mathbb{N}$ & a number without a fractional component in [1, $\infty$) \\
real & $\mathbb{R}$ & any number in (-$\infty$, $\infty$)\\
boolean & $\mathbb{B}$ & true or false \\
\bottomrule
\end{tabular} 
\end{center}

\noindent
The specification of \progname \ also uses some derived data types: 
\begin{itemize}
  \item \textbf{Array} (sequence): lists filled with elements of the same data type
  \item \textbf{Record}: a collection of fields, where each field is a key-value pair
  \item \textbf{String} (\textbf{str}): a sequence of characters 
\end{itemize}

\subsection{Variable Name Traceability}
To help program development and understanding, some notations regarding the eddy profile from the SRS TM and DD are altered in the MIS to better reflect their usage in the code:
\begin{itemize}
  \item intensity: magnitude of the eddy intensity vector ($\boldsymbol{\alpha}$) [SRS: DD3].
  \item orientation: unit vector direction of the eddy intensity vector ($\boldsymbol{\alpha}$) [SRS: DD3].
\end{itemize}

\subsection{Abstract Data Types}

As several modules listed in Section \ref{SecMD} are Abstract Data Types (ADTs), this documents use their types as follows:
\begin{itemize}
  \item \textbf{VectorT}: 3-element NumPy array $\mathbb{R}^3$, representing a 3D position or velocity vector and vector/matrix operation methods [MG: M9].
  \item \textbf{EddyProfileT}: Eddy profile object, stores a Record of different types of eddies with their parameters (intensity, length\_scale), and weights for random generation [MIS\ref{mProf}].
  \item \textbf{FlowFieldT}: Flow field object, stores all EddyT objects in that field, with methods for velocity sum calculation, see Flow Field Module [MIS\ref{mFlow}].
  \item \textbf{QueryT}: Query interface object, handles the request to query a field, see Query Interface Module [MIS\ref{mQuery}].
\end{itemize}

Additional Abstract Data Types:
\begin{itemize}
  \item \textbf{Figure}: matplotlib figure object, for visualization.
\end{itemize}

\section{Module Decomposition} \label{SecMD}

The following table is taken directly from the Module Guide document for this project.
\begin{table}[h!]
  \centering
  \begin{tabular}{p{0.3\textwidth} p{0.6\textwidth}}
  \toprule
  \textbf{Level 1} & \textbf{Level 2}\\
  \midrule
  
  {Hardware-Hiding Module} & ~ \\
  \midrule
  
  \multirow{6}{0.3\textwidth}{Behaviour-Hiding Module}
  & Main Control Module\\
  & Query Interface\\
  & Eddy Profile Module\\
  & Flow Field Module\\
  & Eddy Module\\
  & Shape Function Module\\
  \midrule
  
  \multirow{3}{0.3\textwidth}{Software Decision Module} 
  & File I/O Module\\
  & Vector Module\\
  & Visualization Module\\
  \bottomrule

\end{tabular}
\caption{Module Hierarchy}
\label{TblMH}
\end{table}

\newpage
\section{MIS of Main Control Module} \label{mMain} 

\subsection{Module}
\texttt{main}

\subsection{Uses}
\begin{itemize}
\item Query Interface [MIS\ref{mQuery}]
\item Flow Field Module [MIS\ref{mFlow}]
\item Eddy Profile Module [MIS\ref{mEddy}]
\end{itemize}

\subsection{Syntax}

\subsubsection{Exported Constants}
None.

\subsubsection{Exported Access Programs}

\begin{center}
\begin{tabular}{p{2cm} p{4cm} p{2.5cm} p{5cm}}
\hline
\textbf{Name} & \textbf{In} & \textbf{Out} & \textbf{Exceptions} \\
\hline
\texttt{new}  & \texttt{profile\_name} str,\newline\texttt{field\_name} str,\newline\texttt{dimensions} $\mathbb{R}^3$,\newline\texttt{avg\_vel} $\mathbb{R}$ & - & unrecognized arguments,\newline required args not inputted\\
\texttt{query}  & \texttt{field\_name} str,\newline\texttt{query\_name} str,\newline\texttt{shape\_func\_name} str,\newline\texttt{cutoff} $\mathbb{R}$ & - & unrecognized arguments,\newline required args not inputted\\
\hline
\end{tabular}
\end{center}

\subsection{Semantics}

\subsubsection{State Variables}
\begin{itemize}
  \item \texttt{profile}: EddyProfileT, the eddy profile object to be used for generating the flow field.
  \item \texttt{field}: FlowFieldT, the flow field object to be generated.
  \item \texttt{query}: QueryT, the query interface object to handle the request.
\end{itemize}

\subsubsection{Environment Variables}
\begin{itemize}
  \item \texttt{console}: str, the command line console display.
\end{itemize}

\subsubsection{Assumptions}
None.

\subsubsection{Access Routine Semantics}

\noindent \texttt{new(profile\_name, field\_name, dimensions, avg\_vel)}:
\begin{itemize}
  \item transition:
  \begin{itemize}
    \item \texttt{profile} := \texttt{eddy\_profile.load(profile\_name)} load the eddy profile.
    \item \texttt{field} := \texttt{flow\_field.init(profile, field\_name, dimensions, avg\_vel)} create a new field.
    \item Call \texttt{flow\_field.save()} to save itself.
    \item \texttt{console} := ``Field (\texttt{field\_name}) generated and saved."
  \end{itemize}
  \item exception: exc := ($\nexists$ \texttt{profile\_name} $\vee$ $\nexists$ \texttt{field\_name}$\vee$ $\nexists$ \texttt{dimensions}$\Rightarrow$ ``Required arguments not inputted"). \texttt{avg\_vel} has default value of 0.0 so it can be omitted.
  \item exception: exc := (input $\notin$ \{\texttt{profile\_name}, \texttt{field\_name}, \texttt{dimensions}, \texttt{avg\_vel}\} $\Rightarrow$ ``Unrecognized arguments")
\end{itemize}

\noindent \texttt{query(field\_name, query\_name, shape\_func\_name, cutoff)}:
\begin{itemize}
  \item transition:
  \begin{itemize}
    \item \texttt{field} := \texttt{flow\_field.load(field\_name)} load a saved field.
    \item \texttt{query} := \texttt{query.init(field)} initialize the query interface with the field.
    \item if \texttt{shape\_func\_name} is passed, call \texttt{shape\_func\_nametion.set\_active(shape\_func\_name)} to set the active shape function.
    \item if \texttt{cutoff} is passed, call \texttt{shape\_func\_nametion.set\_cutoff(cutoff)} to set the cutoff value.
    \item \texttt{console} := \texttt{query.handle\_request(query\_name)} to get the velocity vectors at the queried positions and times specified in \texttt{query\_name}, and output an operation summary (where the raw result and plot are saved).
  \end{itemize}
  \item exception: exc := ($\nexists$ \texttt{field\_name} $\vee$ $\nexists$ \texttt{query\_name} $\Rightarrow$ ``Required arguments not inputted"). 
  \item exception: exc := (input $\notin$ \{\texttt{field\_name}, \texttt{query\_name}, \texttt{shape\_func\_name}, \texttt{cutoff}\} $\Rightarrow$ ``Unrecognized arguments")
\end{itemize}

\subsubsection{Local Functions}

None.


\newpage
\section{MIS of Query Interface Module} \label{mQuery} 
% \wss{Use labels for cross-referencing}

% \wss{You can reference SRS labels, such as R\ref{R_Inputs}.}

% \wss{It is also possible to use \LaTeX for hypperlinks to external documents.}

\subsection{Module}
\texttt{query}
% \wss{Short name for the module}

\subsection{Uses}
\begin{itemize}
\item Flow Field Module [MIS\ref{mFlow}]
\item Visualization Module [MIS\ref{mVisual}]
\item File I/O Module [MIS\ref{mFile}]
\end{itemize}

\subsection{Syntax}

\subsubsection{Exported Constants}
None.

\subsubsection{Exported Access Programs}

\begin{center}
\begin{tabular}{p{3cm} p{4cm} p{4cm} p{4cm}}
\hline
\textbf{Name} & \textbf{In} & \textbf{Out} & \textbf{Exceptions} \\
\hline
\texttt{init}& \texttt{field} FlowFieldT & - & - \\
\texttt{handle\_request }& \texttt{query\_name} str & \texttt{response} str & \texttt{InvalidRequest} \\
\hline
\end{tabular}
\end{center}

\subsection{Semantics}

\subsubsection{State Variables}
\begin{itemize}
  \item \texttt{field}: FlowFieldT, the flow field object to be queried.
  \item \texttt{request}: Record from request File
  \begin{itemize}
    \item \texttt{mode}: str, query mode, (``meshgrid" or ``points").
    \item \texttt{params}
    \begin{itemize}
      \item (meshgrid mode) \texttt{low\_bounds}: VectorT, lower bound of the meshgrid.
      \item (meshgrid mode) \texttt{high\_bounds}: VectorT, upper bound of the meshgrid.
      \item (meshgrid mode) \texttt{step\_size}: $\mathbb{R}$, step size of the meshgrid.
      \item (meshgrid mode) \texttt{chunk\_size}: $\mathbb{N}$, grid size in each chunk.
      \item (points mode) \texttt{coords}: \{VectorT\}, array of points to query.
      \item \texttt{time}: $\mathbb{R}$, time to query.
    \end{itemize}
    \item \texttt{plot} (only in meshgrid mode to save a slice cross-section plot)
    \begin{itemize}
      \item \texttt{axis}: str, axis perpendicular to plot slice (``x", ``y", ``z").
      \item \texttt{index}: $\mathbb{N}$, index along the axis to get the slice.
      \item \texttt{size}: $\mathbb{N}^2$, pixel size of the saved image.
    \end{itemize}
  \end{itemize}
  \item \texttt{velocities}: array of VectorT velocities at the queried positions.
  \item \texttt{figure}: figure object outputted by Visualization Module (if requested).
\end{itemize}

\subsubsection{Environment Variables}
\begin{itemize}
  \item \texttt{query\_file}: JSON file containing the query, named after \texttt{query\_name}.
  \item \texttt{result\_file}: NumPy (.npy) file containing the raw computing result of the query.
  \item \texttt{plot\_file}: PNG file containing the visualizing plot, if requested.
\end{itemize}
\noindent The reason behind using a JSON file for the query is related to the expected use case. For large meshgrid, each run will likely be submitted to a cluster. The JSON file allow users to easily pre-define the query, instead of having to input it in command line each time. The JSON file can also be easily generated by other scripts or programs. Directly passing a JSON string is also supported if this module imported by other programs.

\noindent At current stage, the program does not do much after obtaining the raw result, other than saving it to a file. In the future, more processing and analysis can be added.
% \wss{This section is not necessary for all modules.  Its purpose is to capture
%   when the module has external interaction with the environment, such as for a
%   device driver, screen interface, keyboard, file, etc.}

\subsubsection{Assumptions}
None.
% \wss{Try to minimize assumptions and anticipate programmer errors via
%   exceptions, but for practical purposes assumptions are sometimes appropriate.}

\subsubsection{Access Routine Semantics}

\noindent \texttt{init(field)}:
\begin{itemize}
\item transition: \texttt{field} := as inputted
\end{itemize}

\noindent \texttt{handle\_request(query\_name)}:\\
\noindent Currently, points mode is a placeholder. An array of points is handled as many single point meshgrids.
\begin{itemize}
\item transition:
  \begin{itemize}
    \item \texttt{request} := \texttt{file\_io.read(query\_name)} parse \texttt{query\_file} into a Record.
    \item \texttt{velocities} := \texttt{field.sum\_vel\_mesh(request.params)} get the velocity vectors at the queried meshgrid and time.
    \item \texttt{result\_file} := \texttt{file\_io.write(`results', velocities)} save the result.
    \item \texttt{figure} := \texttt{visualize.plot\_mesh(velocities, plot)} render plot if requested.
    \item \texttt{plot\_file} := \texttt{file\_io.write(`plots', figure)} save the plot if requested.
  \end{itemize}
\item output: \texttt{response} := str, operation summary, where the raw result and plot are saved.
\item exception: exc := (\texttt{request} is not a Record or does not have expected parameters $\Rightarrow$ \texttt{InvalidRequest})
\end{itemize}
\noindent More request methods to be implemented in the future.

% \wss{A module without environment variables or state variables is unlikely to
%   have a state transition.  In this case a state transition can only occur if
%   the module is changing the state of another module.}

% \wss{Modules rarely have both a transition and an output.  In most cases you
%   will have one or the other.}

\subsubsection{Local Functions}

% \wss{As appropriate} \wss{These functions are for the purpose of specification.
%   They are not necessarily something that is going to be implemented
%   explicitly.  Even if they are implemented, they are not exported; they only
%   have local scope.}

None.

\newpage
\section{MIS of Eddy Profile Module} \label{mProf} 

\subsection{Module}
\texttt{eddy\_profile}

\subsection{Uses}
\begin{itemize}
\item File I/O Module [MIS\ref{mFile}]
\end{itemize}

\subsection{Syntax}

\subsubsection{Exported Constants}
None.

\subsubsection{Exported Access Programs}

\begin{center}
\begin{tabular}{p{3.2cm} p{3.5cm} p{5cm} p{3cm}}
\hline
\textbf{Name} & \textbf{In} & \textbf{Out} & \textbf{Exceptions} \\
\hline
\texttt{init} & \texttt{profile\_name} str & - & \texttt{InvalidProfile} \\
% \texttt{save} & \texttt{profile\_name} str & - & -  \\
\texttt{get\_density}& - & \texttt{densities} array of $\mathbb{R+}$ & - \\
\texttt{get\_length\_scale}& - & \texttt{length\_scales} array of $\mathbb{R+}$ & - \\
\texttt{get\_intensity}& - & \texttt{intensities} array of $\mathbb{R+}$ & - \\
\hline
\end{tabular}
\end{center}

\subsection{Semantics}

\subsubsection{State Variables}
\begin{itemize}
  \item \texttt{name}: str, name of the eddy profile (is also filename).
  \item \texttt{variants}: array of Records, each containing \{\texttt{density}, \texttt{length\_scale}, \texttt{intensity}\}
  \begin{itemize}
    \item \texttt{density}: $\mathbb{R+}$, how many eddies in a unit volume.
    \item \texttt{length\_scale}: $\mathbb{R+}$, the length scale ($\sigma$) of the eddy variant.
    \item \texttt{intensity}: $\mathbb{R+}$, the intensity magnitude ($|\alpha|$) of the eddy variant.
  \end{itemize}
  
\end{itemize}

\subsubsection{Environment Variables}
\begin{itemize}
  \item \texttt{profile\_file}: JSON file containing the eddy profile, named after \texttt{profile\_name}.
\end{itemize}

\subsubsection{Assumptions}
None.

\subsubsection{Access Routine Semantics}

\noindent \texttt{init(profile\_name)}:
\begin{itemize}
\item transition:
  \begin{itemize}
    \item \texttt{name} := \texttt{profile\_name}
    \item \texttt{variants} := \texttt{file\_io.read(`profiles', profile\_name)} load the eddy profile from \texttt{profile\_file}.
  \end{itemize}
\item exception: exc := (($\nexists \texttt{density} \vee \nexists \texttt{length\_scale} \vee \nexists \texttt{intensity}) \forall \texttt{variants} \Rightarrow$ \texttt{InvalidProfile})
\item exception: exc := ($\neg \forall \{\texttt{density}, \texttt{length\_scale}, \texttt{intensity} \in \texttt{variants}\} >0$ $\Rightarrow$ \texttt{InvalidProfile})
% \item output: out := \texttt{encode(\{VectorT\})}, velocity vectors at all the queried points in one JSON string.
\end{itemize}
%   will have one or the other.}

\noindent \texttt{get\_density()}:
\begin{itemize}
\item output: \texttt{densities} := $[\texttt{density} \in \texttt{variants}]$
\end{itemize}

\noindent \texttt{get\_length\_scale()}:
\begin{itemize}
\item output: \texttt{length\_scales} := $[\texttt{length\_scale} \in \texttt{variants}]$
\end{itemize}

\noindent \texttt{get\_intensity()}:
\begin{itemize}
\item output: \texttt{intensities} := $[\texttt{intensity} \in \texttt{variants}]$
\end{itemize}

\subsubsection{Local Functions}
None.


\newpage
\section{MIS of Flow Field Module} \label{mFlow} 

\subsection{Module}
\texttt{flow\_field}

\subsection{Uses}
\begin{itemize}
\item Eddy Module [MIS\ref{mEddy}]
\item Vector Module [NumPy]
\item File I/O Module [MIS\ref{mFile}]
\end{itemize}

\subsection{Syntax}

\subsubsection{Exported Constants}
None.

\subsubsection{Exported Access Programs}

\begin{center}
\begin{tabular}{p{2.3cm} p{4.3cm} p{3.5cm} p{4cm}}
\hline
\textbf{Name} & \textbf{In} & \textbf{Out} & \textbf{Exceptions} \\
\hline
\texttt{init} & \texttt{profile} EddyProfileT, \newline\texttt{field\_name} str, \newline\texttt{dimensions} VectorT, \newline\texttt{avg\_vel} $\mathbb{R}$ & - & \texttt{InvalidDimensions}\newline\texttt{InvalidAvgVelocity}\newline\texttt{EddyScaleTooLarge}\\
\texttt{load} & \texttt{field\_name} str & \texttt{field} FlowFieldT & -  \\
\texttt{save} & - & - & -  \\
% \texttt{save} & \texttt{profile\_name} str & - & -  \\
\texttt{sum\_vel\_mesh}& \texttt{high\_bounds} VectorT,\newline\texttt{low\_bounds} VectorT,\newline\texttt{step\_size} $\mathbb{R+}$,\newline\texttt{chunk\_size} $\mathbb{N}$,\newline\texttt{time} $\mathbb{R}$ & \texttt{velocities}\newline array of VectorT & \texttt{OutOfBoundary}\newline\texttt{InvalidStepSize}\newline\texttt{InvalidChunkSize}\newline\texttt{InvalidTime} \\
\hline
\end{tabular}
\end{center}

\subsection{Semantics}

\subsubsection{State Variables}
\begin{itemize}
  \item \texttt{profile}: EddyProfileT, eddy profile to be used to generate the flow field.
  \item \texttt{name}: str, name of the flow field (is also filename).
  \item \texttt{dimensions}: VectorT, size of the flow field, with $x$ being the axial direction, $y$ horizontal and $z$ vertical.
  \item \texttt{avg\_vel}: $\mathbb{R}$, average flow velocity along x-axis.
  \item \texttt{N}: $\mathbb{N}$, total number of eddies in the field.
  \item \texttt{init\_x}: array of $\mathbb{R}$, initial $x$-coordinates of all eddies in the field.
  \item \texttt{y}: set \{array of $\mathbb{R}$\}, $y$-coordinates of all eddies in the field at each flow iteration.
  \item \texttt{z}: set \{array of $\mathbb{R}$\}, $z$-coordinates of all eddies in the field at each flow iteration.
  \item \texttt{sigma}: array of $\mathbb{R+}$, length scales ($\sigma$) of all eddies in the field.
  \item \texttt{alpha}: array of VectorT, intensity vector ($\boldsymbol{\alpha}$) of all eddies in the field.
\end{itemize}

\subsubsection{Environment Variables}
\begin{itemize}
  \item \texttt{field\_file}: binary file containing the flow field object, named after \texttt{field\_name}.
\end{itemize}

\subsubsection{Assumptions}
\begin{itemize}
  \item External flow [SRS: A4, MG: AC2]
\end{itemize}

\subsubsection{Access Routine Semantics}

\noindent \texttt{init(profile, field\_name, dimensions, avg\_vel, eddy\_count)}:\\
Initialize the flow field with given eddy profile. Randomly generate eddies based on their parameters and associated weights, and give them initial positions within the flow field.
\begin{itemize}
\item transition: 
  \begin{itemize}
    \item \texttt{profile, name, dimensions, avg\_vel} := as inputted
    \item \texttt{N} := $\sum (\texttt{profile.get\_density()} \times \prod \texttt{dimensions})$
    \item \texttt{init\_x} := [$\texttt{-dimensions(0)/2} \leq \text{random}~\mathbb{R} \leq \texttt{dimensions(0)/2}$] of size \texttt{N}
    \item \texttt{y} := [$\texttt{-dimensions(1)/2} \leq \text{random}~\mathbb{R} \leq \texttt{dimensions(1)/2}$] of size \texttt{N}, for first 3 flow iterations
    \item \texttt{z} := [$\texttt{-dimensions(2)/2} \leq \text{random}~\mathbb{R} \leq \texttt{dimensions(2)/2}$] of size \texttt{N}, for first 3 flow iterations
    \item \texttt{sigma} := array of size \texttt{N}, the sigma (length scale) of each eddy variant is repeated in this array by its density $\times$ field volume.
    \item \texttt{alpha} := array of size \texttt{N}, the magnitude of alpha (intensity) of each eddy variant is repeated in this array by its density $\times$ field volume. Then this array is multiplied by an array of random unit vector to get the alpha vectors of all eddies.
  \end{itemize}
\item exception: 
  \begin{itemize}
    \item exc := (any $d$ $\in$ \texttt{dimensions} $\le$ 0 $\Rightarrow$ \texttt{InvalidDimensions})
    \item exc := (\texttt{avg\_vel} $<$ 0 $\Rightarrow$ \texttt{InvalidAvgVelocity})
    \item exc := (any $2 \times \sigma \in \texttt{sigma}$ $\ge$ any $d \in \texttt{dimensions}$ $\Rightarrow$ \texttt{EddyScaleTooLarge})
  \end{itemize}
\end{itemize}

\noindent \texttt{load(field\_name)}:
\begin{itemize}
  \item output: \texttt{field} := \texttt{file\_io.read(`fields', field\_name)} load the flow field from \texttt{field\_file}.
\end{itemize}

\noindent \texttt{save()}:
\begin{itemize}
  \item transition: \texttt{field\_file} := \texttt{file\_io.write(`fields', field)} save the field object.
\end{itemize}

\noindent \texttt{sum\_vel(low\_bounds, high\_bounds, step\_size, chunk\_size, time)}:\\
This function calculates the velocity at each position within a queried meshgrid, by summing the influence by all nearby eddies at a given time. Due to practical considerations compared to the theoretical models in SRS, it is very hard to explain this part with only formal notations. I had to use nature language.
\begin{itemize}
  \item First, use the queried \texttt{time} to get its corresponding flow iteration and offset. A flow iteration (\texttt{fi}) is defined as when an entire $x$-length of the field dimension has passed due to the average flow velocity in $x$-direction. The offset is the $x$-difference compared to the start of current iteration.
  \item Position array of eddy centers := \texttt{get\_eddy\_centers(fi)}. In this function, if the $y$ and $z$ positions of the input flow iteration is saved in state variables \texttt{y} and \texttt{z}, they will be returned, otherwise they will be randomly generated and stored. Then, the offset is applied to the \texttt{init\_x}. Now the center positions of all eddies at the queried time are obtained.
  \item To satisfy conservation of mass, eddies that are partially outside the field need to be wrapped around to the other side. In the $x$-direction, the previous and next flow iterations are added. In the $y$ and $z$ directions, the current field is copied to outside of each side and diagonally (function \texttt{get\_wrap\_arounds()}). To save computational resources, only the eddies that are within the field or outside but touching the field are kept (function \texttt{within\_margin()}).
  \item The queried region within the field is bounded by \texttt{low\_bounds} and \texttt{high\_bounds}. Using a resolution of \texttt{step\_size}, this region is turned into a meshgrid. Velocity need to be calculated at each point in this meshgrid.
  \item To avoid repeatedly calculating influence by eddies that are far away from any given point, the field is divided into chunks of size \texttt{chunk\_size} in each direction. For each chunk, only eddies that are either inside the chunk or outside but touching the chunk are considered (function \texttt{within\_margin()}).
  \item Call \texttt{eddy.sum\_vel\_chunk()} for each chunk to get the velocity at each point in the chunk. This chunk velocity array is then merged to the entire meshgrid velocity array.
  \item output: \texttt{velocities} := array of VectorT, the velocity vectors at each point in the meshgrid.
\end{itemize}

\subsubsection{Local Functions}

\noindent \texttt{get\_eddy\_centers(flow\_iteration)}:
\begin{itemize}
  \item output: \texttt{centers} := array of VectorT, eddy center positions at the queried flow iteration. See description above.
\end{itemize}

\noindent \texttt{get\_wrap\_arounds()}:
\begin{itemize}
  \item output: \texttt{centers}, \texttt{sigma}, \texttt{alpha} := the center position of each eddy and after wrapping, with its corresponding sigma and alpha. See description above.
\end{itemize}

\noindent \texttt{within\_margin(value, margin, low\_bound, high\_bound)}:
\begin{itemize}
\item output: out := $\mathbb{B}$, $(\texttt{value} < \texttt{high\_bound} + \texttt{margin}) \land (\texttt{value} > \texttt{low\_bound} - \texttt{margin})$
\end{itemize}



\newpage
\section{MIS of Eddy Module} \label{mEddy} 

\subsection{Module}
\texttt{eddy}

\subsection{Uses}
\begin{itemize}
\item Shape Function Module [MIS\ref{mShape}]
\end{itemize}

\subsection{Syntax}

\subsubsection{Exported Constants}
None.

\subsubsection{Exported Access Programs}

\begin{center}
\begin{tabular}{p{2.5cm} p{5cm} p{3cm} p{3cm}}
\hline
\textbf{Name} & \textbf{In} & \textbf{Out} & \textbf{Exceptions} \\
\hline
\texttt{sum\_vel\_chunk} & \texttt{centers} array of VectorT, \newline\texttt{sigma} array of $\mathbb{R+}$, \newline\texttt{alpha} array of VectorT, \newline\texttt{x\_coords} array of $\mathbb{R}$, \newline\texttt{y\_coords} array of $\mathbb{R}$, \newline\texttt{z\_coords} array of $\mathbb{R}$ & \texttt{velocities}\newline array of VectorT & - \\
\hline
\end{tabular}
\end{center}

\subsection{Semantics}

\subsubsection{State Variables}
None.

\subsubsection{Environment Variables}
None.

\subsubsection{Assumptions}
\begin{itemize}
  \item All eddies are spherical (currently the only function available). Additional functions can be added to this library to support oblong/fat eddies. 
\end{itemize}

\subsubsection{Access Routine Semantics}

\noindent \texttt{sum\_vel\_chunk(centers, sigma, alpha, x\_coords, y\_coords, z\_coords)}:\\

\begin{itemize}
\item output: \texttt{velocities} := array of VectorT, the velocity vectors at each point in the chunk due to spherical influence by each eddy. See [SRS: GD1].
\end{itemize}

\subsubsection{Local Functions}

None.


\newpage
\section{MIS of Shape Function Module} \label{mShape}

\subsection{Module}
\texttt{shape\_func\_nametion}

\subsection{Uses}
\begin{itemize}
\item Vector Module [NumPy]
\end{itemize}

\subsection{Syntax}

\subsubsection{Exported Constants}
None.

\subsubsection{Exported Access Programs}
\begin{center}
\begin{tabular}{p{3cm} p{4cm} p{3cm} p{3.5cm}}
\hline
\textbf{Name} & \textbf{In} & \textbf{Out} & \textbf{Exceptions} \\
\hline
\texttt{set\_active} & \texttt{shape\_func\_name} str & - & FunctionNotDefined \\
\texttt{set\_cutoff} & \texttt{cutoff\_value} $\mathbb{R}$ & - & InvalidCutoff \\
\texttt{get\_cutoff} & - & \texttt{cutoff\_value} $\mathbb{R}$ & - \\
% \texttt{save} & \texttt{profile\_name} str & - & -  \\
\texttt{active}& \texttt{dk} $\mathbb{R}$,\newline\texttt{sigma} $\mathbb{R}$ & \texttt{shape\_val} $\mathbb{R}$ & - \\
\texttt{quadratic}& \texttt{dk} $\mathbb{R}$,\newline\texttt{lsigma} $\mathbb{R}$ & \texttt{shape\_val} $\mathbb{R}$ & - \\
\texttt{gaussian}& \texttt{dk} $\mathbb{R}$,\newline\texttt{lsigma} $\mathbb{R}$ & \texttt{shape\_val} $\mathbb{R}$ & - \\
\texttt{...}& \texttt{dk} $\mathbb{R}$,\newline\texttt{sigma} $\mathbb{R}$ & \texttt{shape\_val} $\mathbb{R}$ & - \\
\hline
\end{tabular}
\end{center}
User can modify this module to add more shape functions.

\subsection{Semantics}

\subsubsection{State Variables}
\begin{itemize}
  \item \texttt{active}: the function that is currently designated as the active shape function.
  \item \texttt{cutoff}: $\mathbb{R}$, the cutoff value for the shape function. 
\end{itemize}

\subsubsection{Environment Variables}
None.

\subsubsection{Assumptions}
\begin{itemize}
  \item \texttt{sigma} is always passed as input. Some shape functions may not use it, but it is always assumed to be passed by the caller so that any shape function can be used interchangeably without needing to modify the caller's module.
\end{itemize}

\subsubsection{Access Routine Semantics}

\noindent \texttt{set\_active(shape\_func\_name)}:
\begin{itemize}
\item transition: \texttt{active} := the function in this module with name \texttt{shape\_func\_name},
\newline so that other modules can always call \texttt{shape\_func\_nametion.active()} to use the designated function.
\item exception: exc := (\texttt{shape\_func\_name} $\notin$ \{functions in library\} $\Rightarrow$ \texttt{FunctionNotDefined})
\end{itemize}

\noindent \texttt{set\_cutoff(cutoff\_value)}:
\begin{itemize}
\item transition: \texttt{cutoff} := \texttt{cutoff\_value}
\item exception: exc := (\texttt{cutoff\_value} $\le$ 0 $\Rightarrow$ \texttt{InvalidCutoff})
\end{itemize}

\noindent \texttt{get\_cutoff()}:
\begin{itemize}
\item output: \texttt{cutoff\_value} := \texttt{cutoff}
\end{itemize}

\noindent \texttt{active(dk, sigma)}:
\begin{itemize}
  \item output: \texttt{shape\_val} := $\mathbb{R}$, depending on the currently active function, 
  \newline calls the active shape function with the inputted parameters to get output value.
\end{itemize}

\noindent \texttt{quadratic(dk, sigma)}:
\begin{itemize}
  \item output: \texttt{shape\_val} := $\texttt{sigma}\times(1-\texttt{dk})^2$ if $\texttt{dk} < 1$ otherwise 0. See [SRS: TM1].
  \newline This function is not affected by change of \texttt{cutoff} value, otherwise the output will be negative when \texttt{dk} is greater than 1.
\end{itemize}

\noindent \texttt{gaussian(dk, sigma)}:
\begin{itemize}
  \item output: \texttt{shape\_val} := $Ce^{0.5\pi\texttt{dk}}$ if $\texttt{dk} < 1$ otherwise 0,
  \newline where $C = 3.6276$ (provided by Nikita)
\end{itemize}

\subsubsection{Local Functions}

None.


\newpage
\section{MIS of File I/O Module} \label{mFile} 

\subsection{Module}
\texttt{file\_io}

\subsection{Uses}
\begin{itemize}
\item Hardware Hiding Module [OS]
\end{itemize}

\subsection{Syntax}

\subsubsection{Exported Constants}
None.

\subsubsection{Exported Access Programs}

\begin{center}
\begin{tabular}{p{2cm} p{5cm} p{3cm} p{4cm}}
\hline
\textbf{Name} & \textbf{In} & \textbf{Out} & \textbf{Exceptions} \\
\hline
\texttt{read} & \texttt{sub\_dir} str,\newline\texttt{name} str & Record or Array or FlowFieldT & FailToRead \\
\texttt{write} & \texttt{sub\_dir} str,\newline\texttt{name} str,\newline\texttt{content} Record or Array or FlowFieldT or Figure & - & FailToWrite \\
\hline
\end{tabular}
\end{center}

\noindent This module takes care of file directory (always local to the program), handling specific formats and exceptions, so that other modules can focus on their own tasks.

\subsection{Semantics}

\subsubsection{State Variables}
None.

\subsubsection{Environment Variables}
\begin{itemize}
  \item \texttt{dir}: str, directory of the program
  \item \texttt{file}: file on disk to be read or written
\end{itemize}

\subsubsection{Assumptions}
\begin{itemize}
  \item Array passed is large (such as calculation result), so it is faster to save it in binary format than serializing it to format like JSON.
\end{itemize}


\subsubsection{Access Routine Semantics}

\noindent \texttt{read(sub\_dir, name)}:
\begin{itemize}
\item output: out := Record, Array or FlowFieldT, depending on the type of file read.
\item exception: exc := (file cannot be found or not the expected data type $\Rightarrow$ FailToRead)
\end{itemize}

\noindent \texttt{write(sub\_dir, name, content)}:
\begin{itemize}
\item transition: write the \texttt{content} to the file on disk, depending on data type.
\begin{itemize}
  \item Record $\Rightarrow$ JSON
  \item Array $\Rightarrow$ NumPy binary
  \item FlowFieldT $\Rightarrow$ pickle binary
  \item Figure $\Rightarrow$ PNG
\end{itemize} 
\item exception: exc := (file cannot be written to disk $\Rightarrow$ FailToWrite)
\end{itemize}

\subsubsection{Local Functions}

None.

\newpage
\section{MIS of Visualization Module} \label{mVisual} 

THIS IS A PLACEHOLDER [MG: AC5]

\subsection{Module}
\texttt{visualize}

\subsection{Uses}
\begin{itemize}
\item None
\end{itemize}

\subsection{Syntax}

\subsubsection{Exported Constants}
None.

\subsubsection{Exported Access Programs}

\begin{center}
\begin{tabular}{p{2cm} p{5cm} p{3cm} p{4cm}}
\hline
\textbf{Name} & \textbf{In} & \textbf{Out} & \textbf{Exceptions} \\
\hline
... & ... & ... & ... \\
\hline
\end{tabular}
\end{center}

\subsection{Semantics}

\subsubsection{State Variables}
?

\subsubsection{Environment Variables}
?

\subsubsection{Assumptions}
?

\subsubsection{Access Routine Semantics}
?

\subsubsection{Local Functions}
?


\newpage

\bibliographystyle {plainnat}
\bibliography {../../../refs/References}

\newpage

\section{Appendix} \label{Appendix}

\wss{Extra information if required}

\section{Reflection}

The information in this section will be used to evaluate the team members on the
graduate attribute of Problem Analysis and Design.  Please answer the following questions:

\begin{enumerate}
  \item What are the limitations of your solution?  Put another way, given
  unlimited resources, what could you do to make the project better? (LO\_ProbSolutions)
  \item Give a brief overview of other design solutions you considered.  What
  are the benefits and tradeoffs of those other designs compared with the chosen
  design?  From all the potential options, why did you select the documented design?
  (LO\_Explores)
\end{enumerate}


\end{document}