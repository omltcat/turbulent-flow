\documentclass[12pt, titlepage]{article}

\usepackage{amsmath, mathtools}

\usepackage[round]{natbib}
\usepackage{amsfonts}
\usepackage{amssymb}
\usepackage{graphicx}
\usepackage{colortbl}
\usepackage{xr}
\usepackage{hyperref}
\usepackage{longtable}
\usepackage{xfrac}
\usepackage{tabularx}
\usepackage{float}
\usepackage{siunitx}
\usepackage{booktabs}
\usepackage{multirow}
\usepackage[section]{placeins}
\usepackage{caption}
\usepackage{fullpage}

\hypersetup{
bookmarks=true,     % show bookmarks bar?
colorlinks=true,       % false: boxed links; true: colored links
linkcolor=red,          % color of internal links (change box color with linkbordercolor)
citecolor=blue,      % color of links to bibliography
filecolor=magenta,  % color of file links
urlcolor=cyan          % color of external links
}

\usepackage{array}

\externaldocument{../../SRS/SRS}

%% Comments

\usepackage{color}

\newif\ifcomments\commentstrue %displays comments
%\newif\ifcomments\commentsfalse %so that comments do not display

\ifcomments
\newcommand{\authornote}[3]{\textcolor{#1}{[#3 ---#2]}}
\newcommand{\todo}[1]{\textcolor{red}{[TODO: #1]}}
\else
\newcommand{\authornote}[3]{}
\newcommand{\todo}[1]{}
\fi

\newcommand{\wss}[1]{\authornote{blue}{SS}{#1}} 
\newcommand{\plt}[1]{\authornote{magenta}{TPLT}{#1}} %For explanation of the template
\newcommand{\an}[1]{\authornote{cyan}{Author}{#1}}

%% Common Parts

\newcommand{\progname}{ProgName} % PUT YOUR PROGRAM NAME HERE
\newcommand{\authname}{Team \#, Team Name
\\ Student 1 name
\\ Student 2 name
\\ Student 3 name
\\ Student 4 name} % AUTHOR NAMES                  

\usepackage{hyperref}
    \hypersetup{colorlinks=true, linkcolor=blue, citecolor=blue, filecolor=blue,
                urlcolor=blue, unicode=false}
    \urlstyle{same}
                                


\begin{document}

\title{Module Interface Specification for \progname{}}

\author{\authname}

\date{\today}

\maketitle

\pagenumbering{roman}

\section{Revision History}

\begin{tabularx}{\textwidth}{p{3cm}p{2cm}X}
\toprule {\bf Date} & {\bf Version} & {\bf Notes}\\
\midrule
2024-03-18 & 1.0 & Initial MIS\\
% Date 2 & 1.1 & Notes\\
\bottomrule
\end{tabularx}

~\newpage

\section{Symbols, Abbreviations and Acronyms}

See SRS Documentation at \href{https://github.com/omltcat/turbulent-flow/blob/main/docs/SRS/SRS.pdf}{GitHub repo}

% \wss{Also add any additional symbols, abbreviations or acronyms}

\newpage

\tableofcontents

\newpage

\pagenumbering{arabic}

\section{Introduction}

The following document details the Module Interface Specifications for \progname{}, a software to artificially generate flow field that mimics turbulent flow, which can be use as starting point for CFD simulation.
% \wss{Fill in your project name and description}

Complementary documents include the System Requirement Specifications
and Module Guide.  The full documentation and implementation can be
found at \href{https://github.com/omltcat/turbulent-flow/blob/main/docs/SRS/SRS.pdf}{SRS}, \href{https://github.com/omltcat/turbulent-flow/blob/main/docs/Design/SoftArchitecture/MG.pdf}{MG}.  
% \wss{provide the url for your repo}

\section{Notation}

% \wss{You should describe your notation.  You can use what is below as
  % a starting point.}

The structure of the MIS for modules comes from \citet{HoffmanAndStrooper1995},
with the addition that template modules have been adapted from
\cite{GhezziEtAl2003}.  The mathematical notation comes from Chapter 3 of
\citet{HoffmanAndStrooper1995}.  For instance, the symbol := is used for a
multiple assignment statement and conditional rules follow the form $(c_1
\Rightarrow r_1 | c_2 \Rightarrow r_2 | ... | c_n \Rightarrow r_n )$.

The following table summarizes the primitive data types used by \progname. 

\begin{center}
\renewcommand{\arraystretch}{1.2}
\noindent 
\begin{tabular}{l l p{7.5cm}} 
\toprule 
\textbf{Data Type} & \textbf{Notation} & \textbf{Description}\\ 
\midrule
character & char & a single symbol or digit\\
integer & $\mathbb{Z}$ & a number without a fractional component in (-$\infty$, $\infty$) \\
natural number & $\mathbb{N}$ & a number without a fractional component in [1, $\infty$) \\
real & $\mathbb{R}$ & any number in (-$\infty$, $\infty$)\\
string & str & an array of char\\
\bottomrule
\end{tabular} 
\end{center}

\noindent
The specification of \progname \ uses some derived data types: sequences, strings, and
tuples. Sequences are lists filled with elements of the same data type. Strings
are sequences of characters. Tuples contain a list of values, potentially of
different types. In addition, \progname \ uses functions, which
are defined by the data types of their inputs and outputs. Local functions are
described by giving their type signature followed by their specification.

\section{Module Decomposition}

The following table is taken directly from the Module Guide document for this project.

\begin{table}[h!]
  \centering
  \begin{tabular}{p{0.3\textwidth} p{0.6\textwidth}}
  \toprule
  \textbf{Level 1} & \textbf{Level 2}\\
  \midrule
  
  {Hardware-Hiding Module} & ~ \\
  \midrule
  
  \multirow{6}{0.3\textwidth}{Behaviour-Hiding Module}
  & Query Interface\\
  & Eddy Profile Module\\
  & Flow Field Module\\
  & Eddy Module\\
  & Shape Function Module\\
  & Vector Module\\
  \midrule
  
  \multirow{3}{0.3\textwidth}{Software Decision Module} 
  & Main Control Module\\
  & File I/O Module\\
  & Visualization Module\\
  \bottomrule

\end{tabular}
\caption{Module Hierarchy}
\label{TblMH}
\end{table}

\newpage
~\newpage

\section{MIS of Query Interface} \label{mQuery} 
% \wss{Use labels for cross-referencing}

% \wss{You can reference SRS labels, such as R\ref{R_Inputs}.}

% \wss{It is also possible to use \LaTeX for hypperlinks to external documents.}

\subsection{Module}
\texttt{query}
% \wss{Short name for the module}

\subsection{Uses}
\begin{itemize}
\item Flow Field Module [MIS\ref{mFlow}]
\end{itemize}

\subsection{Syntax}

\subsubsection{Exported Constants}
None.

\subsubsection{Exported Access Programs}

\begin{center}
\begin{tabular}{p{3cm} p{4cm} p{4cm} p{4cm}}
\hline
\textbf{Name} & \textbf{In} & \textbf{Out} & \textbf{Exceptions} \\
\hline
\texttt{handle\_request }& \texttt{request} str & \texttt{response} str & \texttt{InvalidRequestFormat} \\
\hline
\end{tabular}
\end{center}

\subsection{Semantics}

\subsubsection{State Variables}
None.
% \wss{Not all modules will have state variables.  State variables give the module
%   a memory.}

\subsubsection{Environment Variables}
None.
% \wss{This section is not necessary for all modules.  Its purpose is to capture
%   when the module has external interaction with the environment, such as for a
%   device driver, screen interface, keyboard, file, etc.}

\subsubsection{Assumptions}
None.
% \wss{Try to minimize assumptions and anticipate programmer errors via
%   exceptions, but for practical purposes assumptions are sometimes appropriate.}

\subsubsection{Access Routine Semantics}

\noindent \texttt{handle\_request(request)}:
\begin{itemize}
% \item transition: \wss{if appropriate} 
\item output: out := \texttt{encode(\{VectorT\})}, velocity vectors at all the queried points in one JSON string.
\item exception: exc := (\texttt{decode(request)} is not \{VectorT\} $\Rightarrow$ \texttt{InvalidRequestFormat}), the request should be a JSON string of positions vectors to query get the velocity vectors.
\end{itemize}
\noindent More request methods to be implemented in the future.

% \wss{A module without environment variables or state variables is unlikely to
%   have a state transition.  In this case a state transition can only occur if
%   the module is changing the state of another module.}

% \wss{Modules rarely have both a transition and an output.  In most cases you
%   will have one or the other.}

\subsubsection{Local Functions}

% \wss{As appropriate} \wss{These functions are for the purpose of specification.
%   They are not necessarily something that is going to be implemented
%   explicitly.  Even if they are implemented, they are not exported; they only
%   have local scope.}

\noindent \texttt{decode(request)}:
\begin{itemize}
% \item transition: \wss{if appropriate} 
\item output: out := \{VectorT\}, an array of position vectors.
\end{itemize}

\newpage
\section{MIS of Eddy Profile Module} \label{mProf} 

\subsection{Module}
\texttt{eddy\_profile}

\subsection{Uses}
\begin{itemize}
\item File I/O Module [MIS\ref{mFile}]
\end{itemize}

\subsection{Syntax}

\subsubsection{Exported Constants}
None.

\subsubsection{Exported Access Programs}

\begin{center}
\begin{tabular}{p{2.5cm} p{3.5cm} p{5cm} p{4cm}}
\hline
\textbf{Name} & \textbf{In} & \textbf{Out} & \textbf{Exceptions} \\
\hline
\texttt{load} & \texttt{profile\_name} str & - & \texttt{InvalidProfile} \\
% \texttt{save} & \texttt{profile\_name} str & - & -  \\
\texttt{get\_params}& - & \texttt{params} \{{\{$\mathbb{R+}$, $\mathbb{R+}$\}}\} & - \\
\texttt{get\_weights}& - & \texttt{weights} \{$\mathbb{R+}$\} & - \\
\hline
\end{tabular}
\end{center}

\subsection{Semantics}

\subsubsection{State Variables}
\begin{itemize}
  \item \texttt{name}: str, name of the eddy profile (is also filename).
  \item \texttt{params}: \{{\{$\mathbb{R+}$, $\mathbb{R+}$\}}\}, array of strength and radius sets for each type of eddy.
  \item \texttt{weights}: \{$\mathbb{R+}$\}, corresponding array of weights for each type of eddy.
\end{itemize}

\subsubsection{Environment Variables}
None.

\subsubsection{Assumptions}
None.

\subsubsection{Access Routine Semantics}

\noindent \texttt{load(profile\_name)}:
\begin{itemize}
\item transition: \texttt{params, weights} := \texttt{fileIO.read(`profile', profile\_name)}
\item exception: exc := (\texttt{params} is not \{{\{$\mathbb{R+}$, $\mathbb{R+}$\}}\} or \texttt{weights} is not type \{$\mathbb{R+}$\} $\Rightarrow$ \texttt{InvalidProfile})
% \item output: out := \texttt{encode(\{VectorT\})}, velocity vectors at all the queried points in one JSON string.
\end{itemize}
%   will have one or the other.}

\noindent \texttt{get\_params()}:
\begin{itemize}
\item output: out := \texttt{params}
\end{itemize}

\noindent \texttt{get\_weights()}:
\begin{itemize}
\item output: out := \texttt{weights}
\end{itemize}

\subsubsection{Local Functions}
Some functions to compute the eddy profile based on user provided parameters, to be implemented in the future.



\newpage
\section{MIS of Flow Field Module} \label{mFlow} 

\subsection{Module}
\texttt{flow\_field}

\subsection{Uses}
\begin{itemize}
\item Eddy Module [MIS\ref{mEddy}]
\item Vector Module [NumPy]
\item File I/O Module [MIS\ref{mFile}]
\item Visualization Module [MIS\ref{mVisual}] (PLACEHOLDER)
\end{itemize}

\subsection{Syntax}

\subsubsection{Exported Constants}
None.

\subsubsection{Exported Access Programs}

\begin{center}
\begin{tabular}{p{2cm} p{4.5cm} p{3.5cm} p{4cm}}
\hline
\textbf{Name} & \textbf{In} & \textbf{Out} & \textbf{Exceptions} \\
\hline
\texttt{init} & \texttt{profile} EddyProfileT, \newline\texttt{field\_name} str, \newline\texttt{dimensions} VectorT, \newline\texttt{avg\_vel} $\mathbb{R}$, \newline\texttt{eddy\_count} $\mathbb{Z}$ & - & \texttt{InvalidDimensions}\newline\texttt{InvalidAvgVelocity}\newline\texttt{EddyScaleTooLarge}\\
\texttt{load} & \texttt{field\_name} str & - & -  \\
\texttt{save} & - & \texttt{field} Record & -  \\
% \texttt{save} & \texttt{profile\_name} str & - & -  \\
\texttt{get\_vel}& \texttt{position} VectorT,\newline\texttt{time} $\mathbb{R}$ & \texttt{velocity} VectorT & \texttt{OutOfBoundary}\newline\texttt{NegativeTime} \\
\hline
\end{tabular}
\end{center}

\subsection{Semantics}

\subsubsection{State Variables}
\begin{itemize}
  \item \texttt{profile}: eddy\_profileT, eddy profile to be used to generate the flow field.
  \item \texttt{name}: str, name of the flow field (is also filename).
  \item \texttt{dimensions}: Vector T of \{$L_x, L_y, L_z$\}, size of the flow field, with x being the axial direction, y vertical and z horizontal.
  \item \texttt{avg\_vel}: $\mathbb{R}$, average flow velocity along x-axis.
  \item \texttt{eddies}: \{EddyT\}, array of eddies in the flow field.
\end{itemize}

\subsubsection{Environment Variables}
None.

\subsubsection{Assumptions}
\begin{itemize}
  \item External flow [SRS: A4, MG: AC2]
\end{itemize}

\subsubsection{Access Routine Semantics}

\noindent \texttt{init(profile, field\_name, dimensions, avg\_vel, eddy\_count)}:
\begin{itemize}
\item transition: 
  \begin{itemize}
    \item \texttt{profile, name, length\_x, length\_y, length\_z, avg\_vel} := as inputted
    \item \texttt{eddies} := \{\texttt{eddy\_count} number of EddyT randomly generated with \texttt{profile}\}
  \end{itemize}
\item exception: 
  \begin{itemize}
    \item exc := (any of d in \texttt{dimensions} $\le$ 0 $\Rightarrow$ \texttt{InvalidDimensions})
    \item exc := (\texttt{avg\_vel} $<$ 0 $\Rightarrow$ \texttt{InvalidAvgVelocity})
    \item exc := (any of 2$*$\{radius in \texttt{profile}\} $\ge$ any of d in \texttt{dimensions} $\Rightarrow$ \texttt{EddyScaleTooLarge})
  \end{itemize}
\end{itemize}

\noindent \texttt{load(field\_name)}:
\begin{itemize}
  \item transition: all state variables reconstructed from \texttt{fileIO.read(`field', field\_name)}.
\end{itemize}

\noindent \texttt{save(new\_name?)}:
\begin{itemize}
  \item transition: \texttt{name} := \texttt{new\_name} if provided, otherwise name remains the same.
  \item output: out := Record (dict) with all current state variables. Each EddyT in \texttt{eddies} is also converted to a Record. So that \texttt{fileIO.write(`field', Record)} can be called to serialized the current state of the flow field into a JSON string for saving.
\end{itemize}

\noindent \texttt{get\_vel(position, time)}:
\begin{itemize}
  \item output: out := $\sum$ \{\texttt{eddy.get\_vel(position - eddy\_pos(eddy, time))} for each \texttt{eddy} in \texttt{eddies}\}, with wrap-around applied for eddies near the boundary.
\end{itemize}

\subsubsection{Local Functions}


\noindent \texttt{eddy\_pos(eddy, time)}:
\begin{itemize}
\item output: out := VectorT: \{\texttt{\\
  get\_offset(time) + eddy.get\_init\_x(),\\
  eddy.get\_y(get\_iter(time), dimensions[1]),\\
  eddy.get\_z(get\_iter(time), dimensions[2])}\\
\}the center position of the eddy at a given time (moving due to flow).
\end{itemize}

\noindent \texttt{get\_iter(time)}:
\begin{itemize}
\item output: out := \texttt{round($\frac{\texttt{avg\_vel}*\texttt{time}}{\texttt{dimensions[0]}}$)}, the number of iterations of the flow field at a given time (how many x-lengths have passed due to average flow velocity).
\end{itemize}

\noindent \texttt{get\_offset(time)}:
\begin{itemize}
\item output: out := $\texttt{avg\_vel}*\texttt{time} \% \texttt{dimensions[0]}$, the x-offset of the flow field at given time in current iteration.
\end{itemize}


\newpage
\section{MIS of Eddy Module} \label{mEddy} 

\subsection{Module}
\texttt{eddy}

\subsection{Uses}
\begin{itemize}
\item Shape FUnction Module [MIS\ref{mShape}]
\end{itemize}

\subsection{Syntax}

\subsubsection{Exported Constants}
None.

\subsubsection{Exported Access Programs}

\begin{center}
\begin{tabular}{p{2cm} p{5.6cm} p{3cm} p{3cm}}
\hline
\textbf{Name} & \textbf{In} & \textbf{Out} & \textbf{Exceptions} \\
\hline
\texttt{init} & \texttt{field\_dimensions} VectorT, \newline\texttt{strength} $\mathbb{R+}$, \newline\texttt{radius} $\mathbb{R+}$, \newline\texttt{orient} VectorT & - & - \\
% \texttt{save} & \texttt{profile\_name} str & - & -  \\
\texttt{get\_init\_x}& - & \texttt{init\_x} $\mathbb{R}$ & - \\
\texttt{get\_y}& \texttt{iter} $\mathbb{Z}$,\newline\texttt{length\_y} $\mathbb{R+}$ & \texttt{y} $\mathbb{R}$ & - \\
\texttt{get\_z}& \texttt{iter} $\mathbb{Z}$,\newline\texttt{length\_z} $\mathbb{R+}$ & \texttt{z} $\mathbb{R}$ & - \\
\texttt{get\_vel}& \texttt{rel\_position} VectorT & \texttt{velocity} VectorT & - \\
\hline
\end{tabular}
\end{center}

\subsection{Semantics}

\subsubsection{State Variables}
\begin{itemize}
  \item \texttt{init\_x}: $\mathbb{R}$, the initial offset from zero x-position.
  \item \texttt{strength}: $\mathbb{R+}$, intensity of the eddy.
  \item \texttt{radius}: $\mathbb{R+}$ for length scale. Velocity outside of \texttt{radius} is always zero.
  \item \texttt{orient}: VectorT, unit vector describing the orientation of the eddy spin axis.
  \item \texttt{y\_arr}: \{$\mathbb{R}$\}, array of y-positions at each flow iteration.
  \item \texttt{z\_arr}: \{$\mathbb{R}$\}, array of z-positions at each flow iteration.
\end{itemize}

\subsubsection{Environment Variables}
None.

\subsubsection{Assumptions}
\begin{itemize}
  \item EddyT objects are not created manually, but generated by calls from \texttt{flow\_field}. Thus, the caller has ensured the validity of all parameters.
\end{itemize}

\subsubsection{Access Routine Semantics}

\noindent \texttt{init(init\_x, strength, radius, orient)}:
\begin{itemize}
\item transition: 
  \begin{itemize}
  \item \texttt{strength, radius, orient} := as inputted
  \item \texttt{init\_x} := \texttt{rand(0, field\_dimensions[0])}
  \item \texttt{y[0], y[1], y[2]} := \{\texttt{rand(0, field\_dimensions[1])}\}
  \item \texttt{z[0], z[1], z[2]} := \{\texttt{rand(0, field\_dimensions[2])}\}
  \end{itemize}
\end{itemize}

\noindent \texttt{get\_init\_x()}:
\begin{itemize}
  \item output: out := \texttt{init\_x}
\end{itemize}

\noindent \texttt{get\_y(iter, length\_y)}:
\begin{itemize}
  \item transition: \texttt{y[iter]} := \texttt{rand(0, length\_y)} if \texttt{y[iter]} does not exist.
  \item output: out :=\texttt{y[iter]}
\end{itemize}

\noindent \texttt{get\_z(iter, length\_z)}:
\begin{itemize}
  \item transition: \texttt{z[iter]} := \texttt{rand(0, length\_z)} if \texttt{z[iter]} does not exist.
  \item output
  : out :=\texttt{z[iter]}
\end{itemize}
\noindent \texttt{get\_vel(rel\_position)}:
\begin{itemize}
  \item output: out := computed from \texttt{rel\_position} with \texttt{strength, radius, orient} and \texttt{ShapeFunction.active(rel\_position, radius)}, see [SRS: TM1, GM1]
\end{itemize}

\noindent \texttt{dump()}:
\begin{itemize}
\item output: out := Record of all current state variables. Used by \texttt{flow\_field.save()} to serialize the eddy object.
\end{itemize}

\subsubsection{Local Functions}

None.


\newpage
\section{MIS of Shape Function Module} \label{mShape}

\subsection{Module}
\texttt{shape\_function}

\subsection{Uses}
\begin{itemize}
\item Vector Module [NumPy]
\end{itemize}

\subsection{Syntax}

\subsubsection{Exported Constants}
None.

\subsubsection{Exported Access Programs}
\begin{center}
\begin{tabular}{p{2cm} p{5.6cm} p{3cm} p{2cm}}
\hline
\textbf{Name} & \textbf{In} & \textbf{Out} & \textbf{Exceptions} \\
\hline
\texttt{set\_active} & \texttt{active\_func} Function & - & - \\
% \texttt{save} & \texttt{profile\_name} str & - & -  \\
\texttt{active}& \texttt{rel\_position} VectorT,\newline\texttt{radius} $\mathbb{R}$ & \texttt{shape\_val} $\mathbb{R}$ & - \\
\texttt{squared}& \texttt{rel\_position} VectorT,\newline\texttt{radius} $\mathbb{R}$ & \texttt{shape\_val} $\mathbb{R}$ & - \\
\texttt{gaussian}& \texttt{rel\_position} VectorT,\newline\texttt{radius} $\mathbb{R}$ & \texttt{shape\_val} $\mathbb{R}$ & - \\
\texttt{...}& \texttt{rel\_position} VectorT,\newline\texttt{radius} $\mathbb{R}$ & \texttt{shape\_val} $\mathbb{R}$ & - \\
\hline
\end{tabular}
\end{center}
User can modify this module to add more shape functions.

\subsection{Semantics}

\subsubsection{State Variables}
\begin{itemize}
  \item \texttt{active}: The function that is currently designated as the active shape function.
\end{itemize}

\subsubsection{Environment Variables}
None.

\subsubsection{Assumptions}
None.

\subsubsection{Access Routine Semantics}

\noindent \texttt{set\_active(active\_func)}:
\begin{itemize}
\item transition: \texttt{active} := \texttt{active\_func},\newline so that other modules can always call \texttt{shape\_function.active()} to use the designated function. This should be set in \texttt{main} when the program starts.
\end{itemize}

\noindent \texttt{active()}:
\begin{itemize}
  \item output: out := \texttt{init\_x}
\end{itemize}

\noindent \texttt{active(rel\_position, radius)}:
\begin{itemize}
  \item output: out := shape function value, depending on the active shape function.
\end{itemize}

\noindent \texttt{squared(rel\_position, radius)}:
\begin{itemize}
  \item output: out := shape function value computed by taking the distance from the rel\_position to the center \texttt{mag(rel\_position)}, and \texttt{radius} (or length sclae) of the eddy. See [SRS: TM1].
\end{itemize}

\noindent \texttt{gaussian(rel\_position, radius)}:
\begin{itemize}
  \item output: out := Use a different (gaussian) equation to get the above value, as may be preferred by some researchers.
\end{itemize}

\subsubsection{Local Functions}

None.


\newpage
\section{MIS of Main Control Module} \label{mMain} 

\subsection{Module}
\texttt{main}

\subsection{Uses}
\begin{itemize}
\item Query Interface [MIS\ref{mQuery}]
\item Flow Field Module [MIS\ref{mFlow}]
\item Eddy Profile Module [MIS\ref{mEddy}]
\end{itemize}

\subsection{Syntax}

\subsubsection{Exported Constants}
None.

\subsubsection{Exported Access Programs}

\begin{center}
\begin{tabular}{p{2cm} p{5.6cm} p{3cm} p{2cm}}
\hline
\textbf{Name} & \textbf{In} & \textbf{Out} & \textbf{Exceptions} \\
\hline
\texttt{main} & \texttt{command\_args} str, & - & - \\
\hline
\end{tabular}
\end{center}

\subsection{Semantics}

\subsubsection{State Variables}
None.

\subsubsection{Environment Variables}
\begin{itemize}
  \item Command line console.
\end{itemize}

\subsubsection{Assumptions}
None.

\subsubsection{Access Routine Semantics}

\noindent \texttt{main(command\_args)}:
\begin{itemize}
\item \texttt{--new\_field --name <field\_name> --dim <Lx>,<Ly>,<Lz>, --vel <avg\_vel> --count <eddy\_count> --profile <profile\_name>} 
  \begin{itemize}
    \item call \texttt{eddy\_profile.load(profile\_name)} to read the eddy profile file.
    \item call \texttt{flow\_field.init(profile, field\_name, dimensions, avg\_vel, eddy\_count)} to create a new flow field.
  \end{itemize}
\item \texttt{--field <field\_name> --query <request>}
  \begin{itemize}
    \item call \texttt{flow\_field.load(field\_name)} to load the flow field.
    \item call \texttt{query.handle\_request(request)} to get the velocity vectors at the queried points.
  \end{itemize}
\end{itemize}

\subsubsection{Local Functions}

None.


\newpage
\section{MIS of File I/O Module} \label{mFile} 

\subsection{Module}
\texttt{fileIO}

\subsection{Uses}
\begin{itemize}
\item Hardware Hiding Module [OS]
\end{itemize}

\subsection{Syntax}

\subsubsection{Exported Constants}
None.

\subsubsection{Exported Access Programs}

\begin{center}
\begin{tabular}{p{2cm} p{5cm} p{3cm} p{4cm}}
\hline
\textbf{Name} & \textbf{In} & \textbf{Out} & \textbf{Exceptions} \\
\hline
\texttt{read} & \texttt{type} str,\newline\texttt{name} str & Record or Array & FileNotExist \\
\texttt{write} & \texttt{type} str,\newline\texttt{content} Record or Array & - & FailToWrite \\
\hline
\end{tabular}
\end{center}

\subsection{Semantics}

\subsubsection{State Variables}
None.

\subsubsection{Environment Variables}
\begin{itemize}
  \item Files on disk.
\end{itemize}

\subsubsection{Assumptions}
\begin{itemize}
  \item The field name or profile name is the same as the filename.
  \item Saved fields are in \texttt{./fields/} and saved profiles are in \texttt{./profiles/} directories
\end{itemize}


\subsubsection{Access Routine Semantics}

\noindent \texttt{read(type, name)}:
\begin{itemize}
\item output: out := Record or Array, the parsed content of the file.
\item exception: exc := (file cannot be found at \texttt{./<type>/<name>.json} $\Rightarrow$ FileNotExist)
\end{itemize}

\noindent \texttt{write(type, content)}:
\begin{itemize}
\item transition: write the serialized JSON string to the file on disk.
\item exception: exc := (file cannot be written to disk $\Rightarrow$ FailToWrite)
\end{itemize}

\subsubsection{Local Functions}

None.

\newpage
\section{MIS of Visualization Module} \label{mVisual} 

THIS IS A PLACEHOLDER [MG: AC5]

\subsection{Module}
\texttt{visualize}

\subsection{Uses}
\begin{itemize}
\item None
\end{itemize}

\subsection{Syntax}

\subsubsection{Exported Constants}
None.

\subsubsection{Exported Access Programs}

\begin{center}
\begin{tabular}{p{2cm} p{5cm} p{3cm} p{4cm}}
\hline
\textbf{Name} & \textbf{In} & \textbf{Out} & \textbf{Exceptions} \\
\hline
... & ... & ... & ... \\
\hline
\end{tabular}
\end{center}

\subsection{Semantics}

\subsubsection{State Variables}
?

\subsubsection{Environment Variables}
?

\subsubsection{Assumptions}
?

\subsubsection{Access Routine Semantics}
?

\subsubsection{Local Functions}
?


\newpage

\bibliographystyle {plainnat}
\bibliography {../../../refs/References}

\newpage

\section{Appendix} \label{Appendix}

\wss{Extra information if required}

\section{Reflection}

The information in this section will be used to evaluate the team members on the
graduate attribute of Problem Analysis and Design.  Please answer the following questions:

\begin{enumerate}
  \item What are the limitations of your solution?  Put another way, given
  unlimited resources, what could you do to make the project better? (LO\_ProbSolutions)
  \item Give a brief overview of other design solutions you considered.  What
  are the benefits and tradeoffs of those other designs compared with the chosen
  design?  From all the potential options, why did you select the documented design?
  (LO\_Explores)
\end{enumerate}


\end{document}